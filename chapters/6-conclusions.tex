In the previous chapters of this thesis, we presented various approaches to study AD using heterogeneous data, both cross-sectional and longitudinal. In this final chapter, we present a general summary of the contributions described in the thesis, possible clinical applications, and relevant future works that arise from the contributions of the thesis.

\section{Main contributions}
%Be firm in your conclusion just as you were in the introduction.
% Do not use references here
% need to answer the main research question of the thesis
% The conclusion is an opportunity to remind the reader why you took the approach you did, what you expected to find, and how well the results matched your expectations.
Chapter \ref{ch:2-review} contains a review existing research of ML methods for longitudinal data in AD. This review draws a clearer picture of the current research in this field: the main problems addressed, existing methods, and gaps in knowledge. From all the findings of the review, we have focused on three main points: 1) the larger performance detected in methods that incorporated multimodal data, compared to methods that used only one modality; 2) interpretability and reproducibility issues, primarily caused by black box methods and nonpublic datasets; and 3) a lack of unsupervised methods, with supervised methods being restricted to labelled or curated datasets. Our other contributions of the thesis try to address those points: 

\begin{itemize}
\item Chapter \ref{ch:3-cimlr} presents a method that is able to cluster a cohort of AD patients using blood-based markers, with each cluster having an associated blood profile. We then use cross-sectional and longitudinal MRI cortical and subcortical data to study the differences and interactions between those clusters and the disease, finding two different groups characterized by different profiles that presented different presentations of the disease. The method used is an unsupervised approach and we use multimodal data for the analysis. It also is straightforward to interpret, as the nature of the model assigns a weight to each of the features. 
\item Chapter \ref{ch:4-rnnvae} shows a novel method based on recurrent neural networks and variational autoencoders, that is able to model a lower-dimensional subspace from different modalities of medical data, generate missing data, and predict future time points. The approach presented has a high flexibility, being able to combine different modalities, with a variable number of time points, and is also able to capture uncertainty. This method also addresses all three main points of our review: it is a completely unsupervised method, is able to incorporate multimodal, longitudinal data, and due to its generative nature, we can interpret the latent space it generates.
\item Chapter \ref{ch:5-pmhippocampus} showcases an analysis of the relationship between APOE $\varepsilon4$ and age on the hippocampal surface of healthy, non-demented subjects. We segment the hippocampus of all subjects of the cohort to generate 3D surface maps and perform a statistical analysis on age-APOE interactions, comparing our results in the healthy cohort to the results of a second separate cohort containing patients at all stages of the disease. We find similarities between the effects of APOE $\varepsilon4$ in healthy subjects and the effects of the disease on demented subjects, suggesting that APOE $\varepsilon4$ has a similar, milder effect on the hippocampal surface to AD effect. While the analysis approach used in the chapter is a conventional statistical analysis, and it is not directly related to the methods reviewed in Chapter \ref{ch:2-review}, we still answer a relevant question on the early development of AD, linking hippocampal surface, demographic information, and genetics, which could be relevant to develop future studies.
\end{itemize}

% 3. Explain why your work is relevant
\section{Clinical applications}
The contributions described in the previous chapters have all relevant clinical applications and implications. While the work described in this thesis does not have an immediate impact on clinical setting, as it was not a one of its main objectives, there are nonetheless a number of implications for clinical application that ought to be described. \\

% CIMLR
Our approach to the exploration of AD heterogeneity using blood-based biomarkers presented in Chapter \ref{ch:3-cimlr} has shown the potential to uncover hidden subgroups or presentations of the disease using a group of blood-based biomarkers. This has some interesting implications: first, it suggests the potential of distinguishing between different presentations or progression paths of the disease with only noninvasive biomarkers, and this could have important implications in personalized treatment and therapy for patients at all stages of the disease. Second, it can help uncover blood-based biomarkers that capture underlying processes related to the disease that could be candidates for other clinical studies. \\

%% RNNVAE
The reconstruction properties of the multichannel recurrent variational autoencoder proposed in Chapter \ref{ch:4-rnnvae} have also interesting clinical applications: it can provide a personalized way to reconstruct missing acquisitions during patient visits, for example. Moreover, we have also shown that the model has  prediction capacity for future evolution of the patient, together with uncertainty. In a clinical setting, the ability to predict, with an associated uncertainty, the short-term evolution of the patient, could be of use for clinical studies and personalized medicine. \\

Finally, our study on hippocampal surface interaction with APOE and age described in Chapter \ref{ch:5-pmhippocampus} is a more research-focused study, with less direct clinical implications. However, it contributes to our understanding of how APOE $\varepsilon4$ affects the hippocampus of healthy but at risk population, and it could lead to discoveries on the specific effect of APOE$\varepsilon4$ in the brain and why and how it puts the patient at a higher risk of AD.

\section{Future work and research directions}

The contributions presented in this thesis can also help arise new research questions and possible research directions that could lead to new advancements and insights on ML studies on AD. We show, for each chapter, the lines of future work that we believe could be relevant for future studies. \\

Many findings discussed in our longitudinal methods review from Chapter \ref{ch:2-review} have not been completely addressed in this thesis and are still open questions. We expect future ML AD approaches to be designed taking into account prior, existing information of the disease using the deluge of knowledge available, such as biomarker dynamics or risk factors. Methods using this existing information and that could model longitudinal, multimodal data will be extremely valuable to improve the modelling of disease progression. Moreover, given the expansion of AD data initiatives and new cohorts around the world, we believe that future methods will use increasing amounts of data from separate cohorts to improve the reproducibility and generalization of the methods. To solve privacy concerns and other related issues that arise while working on multisite data, techniques like federated learning \cite{Yang2019} are promising.\\

The disease presentations shown in Chapter \ref{ch:3-cimlr} require validation in other cohorts to confirm and expand the possible subtypes of the disease linked to blood-based markers. Further exploration could be made by using longitudinally acquired blood markers, together with whole-brain voxel-based analysis, to find more specific differences between the presentations of the disease. Exploration of blood-based markers for AD, either for subtyping or for diagnosis, is a line of research with recent strong results \cite{Cullen2020,Moscoso2020}, and, given their characteristics (rapid and non-invasive acquisition), it could lead to important discoveries in the coming years. \\

The generative model described in Chapter \ref{ch:4-rnnvae}, while very flexible and promising, was only recently developed and there is still potential for improvements in many areas. We could explore multiple changes in the architecture and extra assumptions, like incorporating time between acquisitions, extended hyperparameter and architecture exploration, and adding more data modalities. Moreover, we have not fully explored the uncertainty the model provides, which is important for prediction models in medical settings. \\

Finally, the analysis on the hippocampus presented in Chapter \ref{ch:5-pmhippocampus} could be extended to further validate and interpret the findings. We propose two direct lines of work: analyze the hippocampal shape change over time for the same cohort, to see if temporal deformations also present differences; and conduct the analysis on the hippocampal subfields to discover the specific regions of the hippocampus where the discovered effect is present. Further research on datasets of cognitively healthy subjects such as the one used in the study ought to be explored more to understand the disease at its early stages and reveal its dynamics: for this reason, studies of the effect of APOE $\varepsilon4$ could also be extended to other parts of the brain that could be affected by a similar effects, and studying them using different modalities. New methods would need to be specifically designed to capture the subtle changes provoked by the disease before any cognitive decline, and they could be extremely useful to characterize the dynamics of early AD.