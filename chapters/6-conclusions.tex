In the previous chapters of this thesis, we have presented various approaches to study AD using heterogeneous data, both cross-sectional and longitudinal. In this final chapter, we present a general summary of the contributions described in the thesis, possible clinical applications, and relevant future works that arise from the contributions of the thesis.

\section{Main contributions}

We can summarize this thesis in four main contributions:

\begin{itemize}
\item Chapter \ref{ch:2-review} surveys research on ML methods for longitudinal data in AD. This review portrays a clear picture of the current research in this field: main problems, existing methods, and gaps in knowledge. We highlight three main findings: 1) a larger performance detected in methods when using multimodal data, compared to methods that used only one modality; 2) lack of interpretability and reproducibility, mainly due to black-box learning models and nonpublic datasets; and 3) a lack of unsupervised methods, with supervised methods being restricted to labelled or curated datasets. Our other contributions of the thesis try to address those points.

\item Chapter \ref{ch:3-cimlr} presents an unsupervised method that can cluster a cohort of AD patients using blood-based biomarkers, with each cluster having an associated blood profile. We use cross-sectional and longitudinal MRI cortical and subcortical data to study the differences and interactions between those clusters and AD, and we show two different presentations of the disease characterized by different blood profiles. The method used is an unsupervised approach and we use multimodal data for the analysis. It also is straightforward to interpret, as the nature of the model assigns a weight to each of the features. Characterization of different subtypes of the disease could lead to better personalized decisions for AD treatments.

\item Chapter \ref{ch:4-rnnvae} shows our proposed multi-channel recurrent variational autoencoder for AD progression, which is completely unsupervised. The method models a lower-dimensional subspace from multimodal and longitudinal data, generates missing modalities, and predicts future time points. The approach presented is highly flexible, can deal with different modalities and time points, and is also able to capture uncertainty. Those characteristics address the three points highlighted from Chapter \ref{ch:2-review}, which are important to tackle the challenges associated with longitudinal data analysis.

\item Chapter \ref{ch:5-pmhippocampus} analyses the relationship between APOE $\varepsilon4$ and age on the hippocampal surface of healthy, non-demented subjects. We take advantage of the unique ALFA dataset containing healthy mid-age people with increased risk of AD. We segment the hippocampus of all subjects of the cohort to generate 3D surface maps and perform multivariate statistical analysis on age-APOE interactions, comparing our results in the healthy cohort to the results of a second separate cohort containing patients at all stages of the disease. We find similarities between the effects of APOE $\varepsilon4$ in healthy subjects and the effects of the disease on demented subjects, suggesting that APOE $\varepsilon4$ has a similar, milder effect on the hippocampal surface compared to AD effect. We answer a relevant question on the early development of AD, linking hippocampal surface, demographic information, and genetics, which could be relevant to develop future studies.
\end{itemize}

\section{Clinical applications}
The contributions described in the previous chapters have all relevant clinical applications and implications. Although the work described in this thesis does not have an immediate impact on a clinical setting, it has several implications for clinical application that are described below. \\

% CIMLR
Our approach to the exploration of AD heterogeneity using blood-based biomarkers presented in Chapter \ref{ch:3-cimlr} shows potential to uncover hidden subgroups or presentations of the disease using a group of blood-based biomarkers. This has some relevant implications. First, it suggests the potential of distinguishing between different presentations or progression paths of the disease with only non-invasive biomarkers, which could have important implications in personalized treatment and therapy for patients at all stages of the disease. Second, it can help uncover blood-based biomarkers that capture underlying processes related to the disease that could be candidates for other clinical studies. \\

%% RNNVAE
The reconstruction properties of the multichannel recurrent variational autoencoder proposed in Chapter \ref{ch:4-rnnvae} have also potential clinical applications: it can provide a personalized way to reconstruct missing acquisitions during patient visits, for example. Moreover, we have also shown that the model has potential for clinical trajectory prediction. In a clinical setting, the ability to predict, with associated uncertainty, the short-term evolution of the patient, could be of use for clinical studies and personalized medicine. \\

Finally, our study on hippocampal surface interaction with APOE and age described in Chapter \ref{ch:5-pmhippocampus} focuses on cognitively healthy patients, to discover early signs of the disease. Early characterization is important because it can facilitate early intervention. The study contributes to our understanding of how APOE $\varepsilon4$ affects the hippocampus of healthy, at risk population, and it could lead to discoveries on the specific effect of APOE$\varepsilon4$ in the brain and why and how it puts the patient at a higher risk of AD.

\section{Future work and research directions}

The contributions presented in this thesis put forward new research questions and open the door to novel research directions that could lead to new advancements and insights on ML studies on AD. We show, for each chapter, the lines of future work that we believe could be relevant for future studies. \\

The findings discussed in our longitudinal methods survey in Chapter \ref{ch:2-review} which were not highlighted in the previous section have not been completely addressed in this thesis, and are still open questions: the use of prior knowledge, and dataset and privacy issues. We expect future ML approaches on AD to be designed taking into account existing biological knowledge of the disease, such as biomarker dynamics or risk factors. Methods using this existing information and that could model longitudinal, multimodal data will be extremely valuable to improve the modelling of AD progression. Moreover, given the expansion of AD data initiatives and new cohorts, we believe that future methods should use increasing amounts of data from separate cohorts to improve the reproducibility and generalization of the methods. To solve privacy concerns and other related issues that can arise while working on multisite data, techniques like federated learning \cite{Yang2019} are promising. \\

The disease subtypes shown in Chapter \ref{ch:3-cimlr} linked to blood-based biomarkers require validation in other cohorts to be confirmed and expanded. Further exploration could be made by using longitudinally acquired blood biomarkers, together with whole-brain voxel-based analysis, to find more specific differences between the presentations of the disease. Exploration of blood-based biomarkers for AD, either for subtyping or for diagnosis, is a line of research with recent strong results \cite{Cullen2020,Moscoso2020}, and, given their characteristics (rapid and non-invasive acquisition), it could lead to important discoveries in the coming years. \\

The generative model described in Chapter \ref{ch:4-rnnvae}, while very flexible and promising, was only recently developed and there is still potential for improvements in many aspects. We could explore multiple changes in the architecture and additional assumptions, such as incorporating time between acquisitions, extending hyperparameter and architecture exploration to improve the predictive performance, and adding more data modalities. Moreover, we have not fully explored the uncertainty the model provides, which is important for prediction models in medical settings. \\

Finally, the analysis of the APOE $\varepsilon4$ effect on hippocampal surface presented in Chapter \ref{ch:5-pmhippocampus} could be extended to further validate and interpret the findings. We propose two direct lines of work: to analyze the hippocampal shape change over time for the same cohort, to see if deformation trajectories over time also present differences linked to APOE, and to conduct the analysis on the hippocampal subfields to discover the specific regions of the hippocampus where the discovered effect is present. Further research on datasets of cognitively healthy subjects such as the one used in the study is necessary to deepen the understanding of the disease at its early stages and reveal its dynamics. The effect of APOE $\varepsilon4$ on other parts of the brain that could be affected by similar effects could also be an important line of research. New methods would need to be specifically designed to capture the subtle changes caused by the disease before any cognitive decline, and they could be extremely useful to characterize the dynamics of early AD.